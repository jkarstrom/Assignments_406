\documentclass{article}

\usepackage{amsthm}
\usepackage{float}
\usepackage{graphicx}
\usepackage{amsfonts}
\usepackage{amsmath}
\usepackage{amssymb}
\usepackage{fullpage}
\usepackage[usenames]{color}
\usepackage{hyperref}
  \hypersetup{
    colorlinks = true,
    urlcolor = blue,       % color of external links using \href
    linkcolor= blue,       % color of internal links
    citecolor= blue,       % color of links to bibliography
    filecolor= blue,        % color of file links
    }

\usepackage{listings}

\definecolor{dkgreen}{rgb}{0,0.6,0}
\definecolor{gray}{rgb}{0.5,0.5,0.5}
\definecolor{mauve}{rgb}{0.58,0,0.82}

\lstset{frame=tb,
  language=haskell,
  aboveskip=3mm,
  belowskip=3mm,
  showstringspaces=false,
  columns=flexible,
  basicstyle={\small\ttfamily},
  numbers=none,
  numberstyle=\tiny\color{gray},
  keywordstyle=\color{blue},
  commentstyle=\color{dkgreen},
  stringstyle=\color{mauve},
  breaklines=true,
  breakatwhitespace=true,
  tabsize=3
}

\theoremstyle{theorem}
   \newtheorem{theorem}{Theorem}[section]
   \newtheorem{corollary}[theorem]{Corollary}
   \newtheorem{lemma}[theorem]{Lemma}
   \newtheorem{proposition}[theorem]{Proposition}
\theoremstyle{definition}
   \newtheorem{definition}[theorem]{Definition}
   \newtheorem{example}[theorem]{Example}
\theoremstyle{remark}
  \newtheorem{remark}[theorem]{Remark}


\title{CPSC-406 Report}
\author{Joan Karstrom \\ Chapman University}

\date{\today}

\begin{document}

\maketitle

\begin{abstract}
Short  summary of purpose and content.
\end{abstract}

\tableofcontents

\section{Introduction}\label{intro}

\section{Homework}\label{homework}

This section contains solutions to homework.

\subsection{Week 2}

Exercise 2.3.2 - Convert to a DFA the following NFA

\begin{table}[h!] % h! uses the float package (https://www.overleaf.com/learn/latex/Positioning_images_and_tables#Positioning_tables)
\centering
\begin{tabular}{r||l|l}
    & 0 & 1 \\
    \hline
    $\rightarrow$ p & $\{q,s\}$ & $\{q\}$ \\
    *q & $\{r\}$ & $\{q,r\}$ \\
    r & $\{s\}$ & $\{p\}$ \\
    *s & $\emptyset$ & $\{p\}$
\end{tabular}
\end{table}

The variables we have are $\Sigma$, $q_0$, Q and F. We need to find $\delta$.
\begin{enumerate}
    \item[$\Sigma$:] $\{0,1\}$
    \item[$q_0$:] p
    \item[Q:] $\{p,q,r,s\}$
    \item[F:] $\{q,s\}$
\end{enumerate}

In order to make this NFA to DFA, we have to draw up a new table now, including the complete subset construction.

\begin{table}[h!] % h! uses the float package (https://www.overleaf.com/learn/latex/Positioning_images_and_tables#Positioning_tables)
\centering
\begin{tabular}{r||l|l}
    & 0 & 1 \\
    \hline
    $\emptyset$ & $\emptyset$ & $\emptyset$ \\
    $\rightarrow$ $\{p\}$ & $\{*q, *s\}$ & $\{*q\}$ \\
    $\{*q\}$ & $\{r\}$ & $\{*q, r\}$ \\
    $\{r\}$ & $\{*s\}$ & $\{p\}$ \\
    $\{*s\}$ & $\emptyset$ & $\{p\}$ \\

    $\{p, *q\}$ & $\{*q, r, *s\}$ & $\{*q, r\}$ \\
    $\{p, r\}$ & $\{*q, *s\}$ & $\{p, *q\}$ \\
    $\{p, *s\}$ & $\{*q, *s\}$ & $\{p, *q\}$ \\

    $\{*q, r\}$ & $\{r, *s\}$ & $\{p, *q, r\}$ \\
    $\{*q, *s\}$ & $\{r\}$ & $\{p, *q, r\}$ \\

    $\{r, *s\}$ & $\{*s\}$ & $\{p\}$ \\

    $\{p, *q, r\}$ & $\{*q, r, *s\}$ & $\{p, *q, r\}$ \\
    $\{p, *q, *s\}$ & $\{*q, r, *s\}$ & $\{p, *q, r\}$ \\
    $\{p, r, *s\}$ & $\{*q, *s\}$ & $\{p, *q\}$ \\
    $\{*q, r, *s\}$ & $\{r, *s\}$ & $\{p, *q, r\}$ \\

    $\{p, *q, r, *s\}$ & $\{*q, r, *s\}$ & $\{p, *q, r\}$ \\
\end{tabular}
\begin{align*}
figure:2.1.1
\end{align*}
\end{table}

In an NFA, there are $N$ states from $2^N$ subsets in DFA. Since we have four states given to us, we have $2^4$ subsets. To organize our new sets, we give them new labels to better keep track of them all.

\begin{table}[h!] % h! uses the float package (https://www.overleaf.com/learn/latex/Positioning_images_and_tables#Positioning_tables)
\centering
\begin{tabular}{c|c}
    New Label & Set \\
    \hline
    A & $\{p\}$ \\
    *B & $\{*q\}$ \\
    C & $\{r\}$ \\
    *D & $\{*s\}$ \\
    E & $\{p, *q\}$ \\
    F & $\{p, r\}$ \\
    G & $\{p, *s\}$ \\
    H & $\{*q, r\}$ \\
    I & $\{*q, *s\}$ \\
    J & $\{r, *s\}$ \\
    K & $\{p, *q, r\}$ \\
    L & $\{p, *q, *s\}$ \\
    M & $\{p, r, *s\}$ \\
    N & $\{*q, r, *s\}$ \\
    O & $\{p, *q, r, *s\}$ \\
\end{tabular}
\begin{align*}
figure:2.1.2
\end{align*}
\end{table}

\pagebreak
To keep it simple, the $\emptyset$ will still be $\emptyset$ in our translation.
The new $\Sigma$, $q_0$, Q, and F are listed below.
\begin{enumerate}
    \item[Q:] $\{A, B, C, D, E, F, G, H, I, J, K, L, M, N, O\}$
    \item[$\Sigma$:] $\{0,1\}$
    \item[$q_0$:] A
    \item[F:] $\{B,D\}$
\end{enumerate}

Using the new labels in figure 2.1.2, we replace what we see in 2.1.1.

\begin{table}[h!] % h! uses the float package (https://www.overleaf.com/learn/latex/Positioning_images_and_tables#Positioning_tables)
\centering
\begin{tabular}{r||c|c}
    & 0 & 1 \\
    \hline
    $\rightarrow$ A & I & *B \\
    *B & C & H \\
    C & *D & A \\
    *D & $\emptyset$ & A \\
    E & N & H \\
    F & I & E \\
    G & I & E \\
    H & J & K \\
    I & C & K \\
    J & *D & A \\
    K & N & K \\
    L & N & K \\
    M & I & E \\
    N & J & K \\
    O & N & K \\
\end{tabular}
\begin{align*}
figure:2.1.3
\end{align*}
\end{table}


\pagebreak
\medskip\noindent
To better understand this using a graphical depiction, this is a map of the $\delta$.

\begin{figure}[h!]
    \centering
    \includegraphics[width=\textwidth]{week2.png}
    \caption{Graphical Depiction - Week 2 Example}
    \label{fig:Graphical Depiction - Week 2 Example}
\end{figure}

\subsection{Week 3}
Question 1: Write down the steps taken by the unification algorithm for each
of the following pairs of terms. If the algorithm succeeds, then write down the
MGU and the corresponding common instance.
\begin{figure}[h!]
    \centering
    \includegraphics[width=\textwidth]{week3.1.jpeg}
    \caption{Unification Algorithm - Week 3 Question 1}
    \label{fig:Unification Algorithm - Week 3 Question 1}
\end{figure}
\pagebreak

Question 2: Consider the following variant of the network connection problems.
\begin{figure}[h!]
    \centering
    \includegraphics[width=\textwidth]{week3.3.png}
    \label{fig:SLD Tree - Week 3 Question 2}
\end{figure}
\begin{figure}[h!]
    \centering
    \includegraphics[width=\textwidth]{week3.2.jpeg}
    \caption{SLD Tree - Week 3 Question 1}
    \label{fig:SLD Tree - Week 3 Question 2}
\end{figure}

\subsection{Week 4}

\ldots

\section{Paper}

...

\section{Conclusions}\label{conclusions}

(approx 400 words) A critical reflection on the content of the course. Step back from the technical details. How does the course fit into the wider world of software engineering? What did you find most interesting or useful? What improvements would you suggest?

\begin{thebibliography}{99}
\bibitem[ALG]{Alg} \href{https://github.com/alexhkurz/algorithm-analysis-2023}{Algorithm Analysis}, Chapman University, 2023.
\end{thebibliography}

\end{document}
